\section{线性回归}

\paragraph{最小二乘法}
\begin{equation}
    \hat{y} = \arg \min_y \sum_i (y_i - y)^2 = \bar{y},
\end{equation}

\paragraph{最大似然估计} 似然函数:$p(x|\vartheta)$是$\vartheta$的函数,因为iid,整体的似然函数为$\prod_i p(y_i|\mu, \sigma^2) = \prod_i \frac{1}{\sqrt{2\pi\sigma^2}}\exp\left(-\frac{(y_i-\mu)^2}{2\sigma^2}\right)$。找到最大的整体的似然:
$$\hat{\mu} = \arg\max_\mu\prod_i p(y_i|\mu, \sigma^2) = \bar{y},$$这和最小二乘一样!

\paragraph{有偏/无偏估计} 方差$\sigma^2$的有偏估计$\hat{\sigma^2}=\frac{1}{N}\sum_{i=1}^N(y_i-\hat{\mu})^2$,无偏估计$\tilde{\sigma^2}=\frac{1}{N-1}\sum_{i=1}^N(y_i-\hat{\mu})^2$

\paragraph{线性回归}
二维情况:$$\arg\min_{a,b}\sum_i(y_i-(ax_i + b))^2$$
也就是代入$y=ax_i + b$,其实在最大似然估计式中代入$\mu=ax_i + b$也可。

\paragraph{求解有约束优化问题} 考虑一个优化问题:
$$\min_x f(x),\text{ subject to }g(x) = 0, h(x) \le 0,$$
用拉格朗日乘数法求解:
$$L(x,\lambda,\eta)=f(x)+\lambda g(x)+\eta h(x),$$
考虑对偶函数:
$$d(\lambda,\eta)=\min_xL(x,\lambda,\eta),$$
当$\eta \ge 0$,对偶函数是原问题的下界。

$$\max_{\lambda,\eta} d(\lambda,\eta)=\max_{\lambda,\eta}\min_xL(x,\lambda,\eta),$$
$\text{ subject to }\eta>0$,必定有$d^*\le f^*$(弱对偶),但是只有是凸优化且满足KKT条件才有强对偶$d^* = f^*$:
$$\text{KKT cond.} \left\{\begin{array}{l}
\nabla f + \lambda\nabla g + \eta\nabla h = 0, \\
g(x) = 0, \\
h(x) \le 0, \\
\eta \ge 0, \\
\eta h(x) = 0,
\end{array} \right.$$

\paragraph{凸优化} 凸集$C$满足$\forall x,y \in C, \forall \alpha \in [0,1]$:
$$\alpha x + (1-\alpha)y \in C,$$
凸函数$f$是定义在凸集$C$的函数,满足$\forall x,y \in C, \forall \alpha \in [0,1]$:
$$f(\alpha x + (1-\alpha)y) \le \alpha f(x) + (1-\alpha)f(y),$$
仿射函数既是凸也是凹函数。凸优化就是在凸集上最小化一个凸函数。

在凸优化中,所有的局部最优解都是全局最优解。如果一个函数是严格凸的(上式取$<$时),那么只有一个全局最优解。

对偶问题也是凸优化问题。

\paragraph{正则化} 不信任数据时使用,可以让参数变小,一个例子:$$\arg\min_{a,b}\sum_i(y_i-(ax_i + b))^2,\text{ s.t. }a^2 \le c,$$
这个问题是有约束优化问题。相同问题的无约束形式如下:
$$\arg\min_{a,b}\sum_i(y_i-(ax_i + b))^2+\lambda a^2,$$
根据KKT条件,要么$\hat{a}^2 = c$,要么$\lambda = 0$

从贝叶斯学派观点,先验项就是正则化的手段。先验就是额外的信息(很多统计学家质疑这个)最大后验估计等于最大似然估计加上一些指定的先验项。
$$p(a,b|\{x_i\}, \{y_i\}, \sigma^2) \propto p(a,b) \prod_i p(y_i, x_i, a, b, \sigma^2),$$
$$p(a,b) = p(a)p(b), p(a|\sigma_a^2) = \frac{1}{\sqrt{2\pi\sigma_a^2}}\exp(-\frac{a^2}{2\sigma_a^2}),$$
最终结果会规约为正则化系数$\lambda = \sigma^2/\sigma_a^2$的最小二乘。

\paragraph{基函数}
用基函数可以将变量使用非线性方法重新映射,常见的基函数有多项式,高斯,sigmoid。应用基函数之后,可以把回归模型写成:
$$y = \bm{w}^T\bm\phi(\bm{x}),$$
然后用最大似然估计或者最小二乘法可得:
$$\bm w = (\Phi^T\Phi)^{-1}\Phi^T\bm y,$$其中,$\Phi = \begin{bmatrix}
\bm\phi_{|1}(x_1) & \cdots & \bm\phi_{|M}(x_1) \\
\vdots & \ddots & \vdots \\
\bm\phi_{|1}(x_N) & \cdots & \bm\phi_{|M}(x_N)
\end{bmatrix}$是设计矩阵,$(\Phi^T\Phi)^{-1}\Phi^T$是$\Phi$的伪逆阵,$\bm y = [y_1, \ldots, y_N]^T$。

用得比较多的基函数如下:
\begin{enumerate}
    \item 多项式:$\phi_i(x) = x^{i-1}$
    \item 高斯:$\phi_i(x) = \exp\{-\frac{(x-\mu_i)^2}{2\sigma^2}\}$
    \item sigmoid:$\phi_i(x) = \mathrm{sigmoid}(\frac{x - \mu_i}{a})$
\end{enumerate}

\paragraph{核函数}
$$E_D(w)+\lambda E_R(w)=\frac{1}{2}\sum_{i=1}^{N}(y_i-w^T\phi(x_i))^2 + \frac{\lambda}{2}w^Tw$$
求解岭回归得到:
$$w_\mathrm{ridge} = (\Phi^T\Phi + \lambda I)^{-1}\Phi^T y,$$
其中,$\Phi$是design matrix,由所有的基函数和数据样本决定。那么输出:
$$
\begin{aligned}
    \hat{y} &= w_\mathrm{ridge}^T\phi(x) = \phi^T(x)(\Phi^T\Phi + \lambda I)^{-1}\Phi^T y \\
    &= \sum_{i=1}^N \phi^T(x)(\Phi^T\Phi + \lambda I)^{-1}\phi(x_i) y_i\\
    &= \sum_{i=1}^N k(x, x_i)y_i,
\end{aligned}
$$
等价核就是$k(x, x_i)$,是按$x = x_i$对称的函数,允许负数值。

\paragraph{偏差-方差分解}
$\mathbb{E}(\bm y - \hat{\bm w}^T\cdot \phi(\bm x))^2 = \int ({\bm w}^T\cdot \phi(\bm x)) - \hat{\bm w}^T\cdot \phi(\bm x))^2p(x)\mathrm{d}x + \int e^2p(e)\mathrm{d}e,$
第二项是噪声,我们只考虑对第一项进行分析。假设我们已经在数据集$\mathcal{D}$上进行了训练,那么有参数$\hat{\bm w}(\mathcal{D})$。那么$\mathbb{E}_\mathcal{D}[{\bm w}^T\cdot \phi(\bm x) - \hat{\bm w}^T\cdot \phi(\bm x)]^2 = (({\bm w} - {\bm w}^*)\cdot \bm\phi(\bm x))^2 + \mathbb{E}_\mathcal{D}[((\hat{\bm w}(\mathcal{D}) - {\bm w}^*)\cdot \bm\phi(\bm x))^2],$其中,前一项是$(\text{bias})^2$,后一项是variance,${\bm w}^*$是$\mathbb{E}_\mathcal{D}[\hat{\bm w}(\mathcal{D})]$。也就是说:expected ``loss'' = (bias)\textsuperscript{2} + variance + noise。过度正则化的模型有很大的偏差,欠缺正则化的模型则有很大的方差。用交叉验证可以找到合适权衡位置。

\paragraph{不同的正则化形式} 最小二乘法:$$\frac{1}{2}\sum^N_{i=1}(y_i - \bm w^T\bm\phi(\bm x))^2,$$
岭回归:$$
\frac{1}{2}\sum^N_{i=1}(y_i - \bm w^T\bm\phi(\bm x))^2 + \frac{\lambda}{2}\bm w^T\bm w,$$
$L_q$-范数正则化回归:$$
\frac{1}{2}\sum^N_{i=1}(y_i - \bm w^T\bm\phi(\bm x_i))^2 + \frac{\lambda}{2}\sum_{j=1}^M |\bm w_j|^q,$$也就是,$\bm w$的可行域在$\sum_{j=1}^M |\bm w_j|^q < c$中;$q=2$时,是岭回归,$q=1$时,是LASSO回归,有稀疏性,可以取代$q<1$;$q<1$时,可行域不再是凸集,不再是凸优化,有稀疏性;$q=0$时,是稀疏回归,是NP-Hard问题。

\begin{figure}[H]
\centering
\includegraphics[width=0.7\columnwidth]{sl/1.png}
\end{figure}


解LASSO先考虑一个特殊的情况:$\Phi^T\Phi = I$,此时最小二乘的解是$w_\mathrm{LS} = \Phi^Ty$
\begin{equation}
\begin{aligned}
    &\min_w \frac{1}{2}\sum^N_{i=1}(y_i - \bm w^T\bm\phi(\bm x_i))^2 + \lambda\|\bm w\|_1 \\
    &=\! \min_w \frac{1}{2}\sum^N_{i=1}(y_i\! - (w_\mathrm{LS}\! - w_\mathrm{LS}\! +\! \bm w)^T\bm\phi(\bm x_i))^2\! +\! \lambda\|\bm w\|_1 \\
    &\to \min_w \frac{1}{2}(\bm w - w_\mathrm{LS})^2 + \lambda\|\bm w\|_1 \\
\end{aligned} 
\end{equation}

所以解为${w_\mathrm{lasso}}_i = \mathrm{sign}({w_\mathrm{LS}}_i) \max (|{w_\mathrm{LS}}_i| - \lambda, 0)$

理解:Best subset(L0): Hard thresholding,
Ridge(L2): Uniformly shrink,
LASSO(L1): Soft thresholding.

\begin{figure}[H]
\centering
\includegraphics[width=0.7\columnwidth]{sl/lasso.png}
\end{figure}

\paragraph{贝叶斯线性回归} 
先定义先验$p(w) = \mathrm{N}(w|m_0, S_0)$。
似然函数为$p(y_i|x_i,w,\beta^{-1}) = \prod_{i=1}^{N}(y_i|w^T\Phi(x_i),\beta^{-1})$
,其中$\beta^{-1}=\sigma_e^2$。
后验概率为$p(w|x_i,y_i)=N(w|m_N,S_n)$
,其中$m_N=S_N(S_0^{-1}m_0+\beta\Phi^Ty)$,
$S_N^{-1}=S_0^{-1}+\beta\Phi^T\Phi$。

最大后验概率估计:$$w_{MAP}=m_N \to \beta(\alpha I+\beta \Phi^T\Phi)^{-1}\Phi^Ty$$

最大似然估计:
$$w_{ML}=(\Phi^T\Phi)^{-1}\Phi^Ty$$

脊回归:
$$w_{ridge}=(\Phi^T\Phi+\lambda I)^{-1}\Phi^Ty$$

0均值高斯先验的贝叶斯估计等于ridge回归解,0均值Laplace先验的贝叶斯估计等于LASSO回归解。共轭先验:使得后验与先验遵循同样形式的分布。注:共轭指的是先验分布与似然函数。

在贝叶斯框架中,每一个变量都有一个分布,给定一个x值,可以从w空间中取参数w进行预测。拟合曲线取决于基函数,具有一定的函数拟合限制。依赖于数据,样本点近,把握越大。从已知数据中估计后验概率:
$$
\begin{aligned}
    p(y|x) &=\int p(y|x,w)p(w|x_i,y_i)dw \\
    &=\int N(y|w^T\Phi(x),\beta^{-1})N(w|m_N,S_N)dw \\
    &=N(y|m_N\Phi(x),\theta_N^2(x)) \\
\end{aligned}
$$
其中$\theta_N^2(x)=\beta^{-1}+\Phi^T(x)S_N\Phi(x)$。 $S_N$项会随着N增大而消失。

\paragraph{模型选择} 
贝叶斯模型选择:$p(M|D) = \frac{p(D|M)p(M)}{p(D)}$,
ML模型选择:$p(D|M)=\int_{w\in M} p(D|w)p(w)dw$.

\section{线性分类}

\paragraph{分类算法的分类}二分类、多分类、多标签分类(多个二分类的聚合)

\paragraph{分类和回归} 都想研究两个变量间的关系,离散情况就是分类了。分类需要量化,保证离散的输出。如果用普通的回归做分类,需要使用sign函数量化,但是这很难解。

\paragraph{逻辑回归} 使用sigmoid函数$\frac{1}{1+e^{-x}}$替代sign(),易解了很多。需要重新映射$y_i = \frac{t_i + 1}{2}$,并且使用交叉熵函数而不是平方误差之和:$\min_{\bm w, b} \sum_i - y_i\log\hat{y_i} - (1-y_i)\log(1-y_i)。$

\paragraph{交叉熵} 逻辑回归不是回归到特定的类别号,而是回归出一个属于某类的概率。这个概率的似然函数$P(t_i|x_i,\bm w,b) = \hat{y_i},\text{ if }t_i = +1\text{ else if }t_i = -1, 1 - \hat{y_i}$。
可以改写成$\hat{y_i}^{y_i}\cdot(1-\hat{y_i})^{(1-y_i)}$,然后取对数似然即可获得交叉熵。

其他解释:有两个概率分布,一个是ground-truth $P(t)$,一个是predicted $Q(t)$,那么交叉熵$C(P,Q) = \sum_tP(t)(-\log Q(t))$,熵$H(P)=\sum_tP(t)(-\log P(t))$,K-L散度(相对熵,两个分布的差异度,但是不对称$D_{KL}(P||Q) \neq D_{KL}(Q||P)$,$\ge 0$)$D_{KL}(P||Q) = C(P,Q) - H(P) = \sum_tP(t)\log{\frac{P(t)}{Q(t)}}$

\paragraph{解释逻辑回归}逻辑回归的预测值满足:
$$
\begin{array}{ll}
\hat{y_i} &= p(t_i = +1 |\bm x_i) \\
&= \frac{p(x_i|t_i=+1)p(t_i=+1)}{p(x_i|t_i=+1)p(t_i=+1) + p(x_i|t_i=-1)p(t_i=-1)} \\
& = \frac{1}{1+e^{-(\bm w^T\bm x_i + b)}}
\end{array}
$$
$$\bm w^T\bm x_i + b = \ln \frac{p(\bm x_i| t_i = +1)p(t_i = +1)}{p(\bm x_i| t_i = -1)p(t_i = -1)} = \text{log odds},$$
几率,就是发生和不发生的比值,逻辑回归使用了对数几率(log odds)。
假设$p(x|t_i=k)$是高斯分布,且具有相同的方差,均值不同,则得到形式如同sigmoid的表达式:
$$\text{log odds} = \Sigma^{-1}(\mu_{+1}-\mu_{-1})\bm x_i + b$$

\paragraph{指数族} 指概率分布可以写成这样的分布:
$$p(\bm x|\bm \vartheta) = h(\bm x)g(\bm\vartheta)\exp(\bm\vartheta^T\bm\phi(\bm x)),$$
其中,$h(\bm x)$是,$\bm\vartheta$是自然参数,$\bm\phi(\bm x)$是充分统计量(为了估计分布所需要的统计量)。

\begin{table}[H]
\centering
\begin{tabular}{lll}
    \toprule
    分布 & 自然参数 & 充分统计 \\
    \midrule
    伯努利 & $\ln(p/(1-p))$ & $x$ \\
    泊松 & $\ln\lambda$ & $x$ \\
    指数 & $-\lambda$ & $x$ \\
    拉普拉斯 & $-1/b$ & $|x-\mu|$ \\
    \bottomrule
\end{tabular}
\end{table}
累积函数$\bm A(\bm\vartheta) = - \ln \bm g(\bm \vartheta)$,有以下性质:
$$
\begin{array}{l}
    \frac{\partial\bm A}{\partial \vartheta_i} = \mathbb{E}[\bm\phi_i(\bm x)] \\
    \frac{\partial^2\bm A}{\partial \vartheta_i\partial \vartheta_j} = \mathbb{E}[\bm\phi_i(\bm x)\bm\phi_j(\bm x)] - \mathbb{E}[\bm\phi_i(\bm x)]\mathbb{E}[\bm\phi_j(\bm x)]
\end{array}
$$
也就是说偏导数是均值、二阶偏导数是协方差,这样的分布具有最大熵的特性。
例子:正态分布中,$\sigma^2=\mathbb{E}(x^2)-\mathbb{E}^2(x)$

\paragraph{最大熵}
具有最大熵性质的离散分布是均匀分布,具有最大微分熵性质的分布是高斯分布。\\
最大熵: $\max_P = \sum_x P(x)\log P(x)$, subject to $\sum_x P(x)=1$, $F_k=\sum_x f_k(x)P(x)$ \\
Gibbs分布: $P(x)=\frac{1}{Z}exp(-\sum_k \lambda_k f_k(x))$ 是一种离散分布,也属于指数族 \\
最大微分熵: $\int_x -P(x)\log p(x)dx$ 


\paragraph{求解逻辑回归} 其最小二乘的梯度为:
$$\nabla E(\bm w) = \sum_i (\hat{y_i} - y_i)\bm \phi(\bm x_i),$$
所以,没有解析解(封闭解),只能求助于以下的数值算法。

\paragraph{优化问题的数值解法} 
\begin{enumerate}
    \item 二阶:Newton-Raphson
    \item 一阶:Gradient Descent, Frank-Wolfe
\end{enumerate}

牛顿法,由泰勒的二阶展开,并令一阶导数为零可得:
$$x^{\prime} = x - H^{-1}\nabla f(x),$$
其中,$H$是海森矩阵。

梯度下降,由泰勒一阶展开,令一阶导数为零可得:
$$x^{\prime} = x -\eta\nabla{}f(x)$$

Frank-Wolfe,求解约束优化问题,也是一阶泰勒展开,考虑$s_t = \min xf^\prime(x_t)$,找到$s_t$之后,每步找到个系数$\gamma \in (0,1)$:
$$x_{t+1} = \gamma s_t - (1-\gamma)x_t$$

\paragraph{线性判别分析(LDA)} 最大化类之间的间距、最小化类内的方差,这在几何上也行得通。经分析其是高斯-逻辑回归的近似。

\paragraph{感知机} 可以用梯度下降解
$$\min_{w,b} \sum_{i=1}^N (t_i - \mathrm{sign}(\bm w^T\bm x_i + b)).$$
若数据集可以找到线性分类器将其分开,则感知机一定能收敛。

\paragraph{多分类--用二分类模拟}数据集划分时总是有未分类的例子:
1)one-versus-the-rest
2)one-versus-one


\paragraph{多分类--softmax回归} 
和前面解释差不多,只不过:
$$\hat{y_i}_k = p(t_i = k|x_i) = \frac{p(x_i|t_i = k)p(t_i = k)}{\sum_m p(x_i|t_i = m)p(t_i = m)},$$
定义$\ln p(x_i|t_i = k)(t_i = k) = {w_k}^Tx_i + b_k$,这就是softmax回归

\paragraph{多标签分类} 就是多个二分类器连接在一块

\section{SVM}
\paragraph{硬边际SVM} 思想是最大化类之间的边际,这样对噪声不敏感且有最好的泛化能力。找到离分类边界最近的点到边界的距离:
$$\gamma = \min_i \frac{y_i(\bm w^T\bm x_i + b)}{\|\bm w\|}$$,我们使最近的点的$y_i(\bm w^T\bm x_i + b) = 1$(通过缩放系数,这些点叫做支持向量),那么显然需要最大化$1/\|\bm w\|$,也就是:
$$\min_{w,b} \frac{1}{2}\|\bm w\|^2, \text{s.t.} 1 - y_i(\bm w^T\bm x_i + b) \le 0,$$
其拉格朗日函数为:
$$L(\bm w, \bm x, \bm \alpha) = \frac{1}{2}\|\bm w\|^2 - \sum_i \bm \alpha_i(y_i(\bm w^T\bm x_i + b) - 1),$$套KKT,结果如下:
$$\begin{array}{l}
\bm w = \sum_i \alpha_iy_i\bm x_i, \\
\sum_i \alpha_iy_i = 0, \\
\alpha_i \ge 0, \\
y_i(\bm w^T\bm x_i + b) \ge 1, \\
\alpha_i(y_i(\bm w^T\bm x_i + b) - 1) = 0,
\end{array}$$
也就是,要么$\alpha_i = 0$,要么$\bm w^T\bm x_i + b = \pm 1$(此时$\bm x_i$是支持向量)。
\paragraph{对偶问题}
$$\max_\alpha\min_{\bm w, b} \frac{1}{2}\|\bm w\|^2 - \sum_i \bm \alpha_i(y_i(\bm w^T\bm x_i + b) - 1),$$
套KKT,结果如下:
$$
\begin{array}{l}
\max_\alpha \sum_i \alpha_i - \frac{1}{2}\sum_i\sum_j \alpha_i\alpha_jy_iy_j\bm{x}_i^T\bm{x}_j, \\
\text{s.t.} \forall \alpha_i \ge 0, \sum_i \alpha_iy_i = 0 \\
\bm w = \sum_i \alpha_iy_i\bm x_i \\
b = y_i - \bm w^T \bm x_i
\end{array}
$$

\paragraph{软边际SVM} 如果非线性可分,我们应该使用软边际SVM。此时我们会对错误分类的点、还有过于靠近分类面的点进行容忍。
$$\min_{w,b} \frac{1}{2}\|\bm w\|^2, \text{s.t. } \mathrm{Ind}(y_i(\bm w^T\bm x_i + b) - 1) \le c_0,$$
其中,Ind函数是指示函数,仅在小于0时得1。也就是说$y_i(\bm w^T\bm x_i + b) \le 1$时有效。那么可以写成另外一种形式:
$$
\min_{w,b} \frac{1}{2}\|\bm w\|^2 + C\sum_i \xi_i, \text{s.t.}\xi_i \ge 0, y_i(\bm w^T\bm x_i + b) \ge 1 - \xi_i,
$$
其中,$\xi = \max(0, 1-y_i(\bm w^T\bm x_i + b))$,叫做松弛变量。其拉格朗日函数如下:
$$
L(\bm w, \bm x, \xi, \bm \alpha, \bm \beta) = \frac{1}{2}\|\bm w\|^2 - \sum_i \bm \alpha_i(y_i(\bm w^T\bm x_i + b) - 1),
$$
套KKT,结果如下:
$$
\begin{array}{l}
\bm w = \sum_i \alpha_iy_i\bm x_i, \\
\sum_i \alpha_iy_i = 0, \\
\alpha_i + \beta_i = C \Rightarrow 0 \le \alpha_i \le C, \\
\alpha_i \ge 0, \beta_i \ge 0, \xi_i \ge 0, \beta_i\xi_i = 0\\
y_i(\bm w^T\bm x_i + b) \ge 1 - \xi_i, \\
\alpha_i(y_i(\bm w^T\bm x_i + b) - 1 + \xi_i) = 0, 
\end{array}
$$
对偶问题如下:
$$
\begin{array}{l}
\max_\alpha \sum_i\alpha_i - \frac{1}{2} \sum_i\sum_j\alpha_i\alpha_jy_iy_j\bm{x}_i^T\bm{x}_j, \\
\text{s.t. }0 \le \alpha_i \le C, \sum_i \alpha_i y_i = 0,
\end{array}
$$
也就是说,$C = +\infty$时,就是硬边际SVM,样本分为三类,后两类是支持向量:
$$
\left\{
\begin{array}{ll}
\alpha_i = 0, & \xi_i = 0, y_i(\bm w^T\bm x_i + b) > 1 \\
0 < \alpha_i < C, & \xi_i = 0, y_i(\bm w^T\bm x_i + b) = 1 \\
\alpha_i = C, & \xi_i > 0, y_i(\bm w^T\bm x_i + b) < 1
\end{array}
\right.
$$
\paragraph{核技巧}
把上文中所有$\bm x_i$换成$\bm\phi(\bm x_i)$,即可获得基函数版本的SVM。将上文中的$\bm{x}_i^T\bm{x}_j$换成$k(\bm x_i, \bm x_j)$,可以得到核函数版本的SVM。其实核函数满足:
$$k(\bm x_i, \bm x_j) = \bm\phi(\bm x_i)^T\bm\phi(\bm x_j),$$
常见的核函数有线性$\bm x_i^T\bm x_j$、RBF$\exp(-\|\bm x_i - \bm x_j\|^2/2\sigma^2)$、多项式$(1+\bm x_i^T\bm x_j)^p$、sigmoid核$\tanh(\alpha\bm x_i^T\bm x_j+\beta)$。

使用核函数与使用基函数并无什么不同,相反核函数还较基函数容易表示一些。满足Mercer's condition就是核函数,也就是说是必须是正定的:
$$\int_{\bm x}\int_{\bm y}f(\bm x)K(\bm x, \bm y)f(\bm y)\mathrm{d}\bm x\mathrm{d}\bm y > 0, \forall f$$

\paragraph{SMO算法}
序列最小优化算法,每次选择两个拉格朗日乘子作为变量,其他的乘子不变。变量乘子都在“盒限制”中选取新值。是坐标下降,比梯度下降简单。

\begin{figure}[H]
    \centering
\includegraphics[width=0.7\columnwidth]{smo.png}
\end{figure}

\section{监督学习}
前面我们学习了回归、分类、密度估计,这些都是完全的监督学习。
监督学习的基本方法:收集数据(清洗)$\rightarrow$决定模型形式(超参)$\rightarrow$决定策略/优化目标$\rightarrow$模型学习(获取参数)$\rightarrow$评价模型

\paragraph{判别模型和概率模型}
判别模型:$\hat{y} = f(x)$(分类、回归),概率模型$q(y|x)$(分类、回归、密度估计)。如果目标变量是离散的,就是分类问题;是连续的就是回归问题。

\paragraph{判别模型和生成模型}
判别模型:$\hat{y} = f(x)$或者$q(y|x)$,生成模型$\hat{x} = f(y)$或者$q(x|y)$。在监督学习中,生成模型不是必要的。如果两个模型都学习了,我们实际上完成了对$p(x,y)$的估计。密度估计在监督学习中是万能的,但是很难解决。判别模型方法只需要处理判别模型,并且可以很容易运用基/核函数,在小数据集上表现更好。生成模型方法需要处理判别模型和生成模型,可以使用隐变量,在处理大量数据时更加容易拟合。

\paragraph{损失函数} $\mathcal{L}(x_i,y_i,f)$,$f$是泛函,下面是一些例子:
\begin{itemize}
    \item 平方损失:$(y_i - f(x_i))^2$
    \item 绝对损失:$|y_i - f(x_i)|^2$
    \item 0-1损失:$\text{if $y_i = f(x_i)$ then 0 else 1}$
    \item log损失,通常见于概率函数(逻辑回归):$-\log q(y_i|x_i)$
    \item 铰链损失,感知机:$\left\{\begin{aligned}
        0, \ \text{if}\ y_i(w^Tx_i +b) \ge 0, \\
        -y_i(w^Tx_i +b), \text{otherwise}
    \end{aligned}\right.$
    \item 铰链损失,SVM:$\left\{\begin{aligned}
        0, \ \text{if}\ y_i(w^Tx_i +b) \ge 1, \\
        1-y_i(w^Tx_i +b), \text{otherwise}
    \end{aligned}\right.$
\end{itemize}

\paragraph{风险函数}
设定输入空间$x\in \mathcal{X}$,输出空间$y\in \mathcal{Y}$,假设空间$f: \mathcal{X} \to \mathcal{Y}, f(x) \in \mathcal{H}$,参数化$f(x) \to f(x|\alpha)$,参数空间$\alpha \in \Lambda$。那么风险函数就是损失函数的函数$R: \Lambda \to \mathbb{R}$:
$$R(\alpha) = \int_{x\in \mathcal{X}, y\in \mathcal{Y}} L(x, y, \alpha)p(x,y) \mathrm{d}x\mathrm{d}y$$
风险最小化就是我们想要的,但是很难进行估计。

\paragraph{经验风险} 基于训练数据,我们用经验风险估计风险:
$$R_\text{emp}(\alpha) = \frac{1}{N} \sum^N_{i=1}L(x_i, y_i, \alpha),$$
其实最小二乘法、最大似然估计就是经验风险最小化(ERM)。有很多关于ERM的问题:
ERM可能会导致不适定问题,因此我们使用正则化。ERM 不能包含先验信息,因此我们使用贝叶斯。ERM不考虑数据集方差,因此我们使用偏差-方差权衡。ERM没有考虑模型存储的成本,所以我们使用
最小描述长度 (MDL)。ERM能保证风险最小化吗? 

\paragraph{过拟合} 原因如下:
\begin{itemize}
    \item 训练数据不足
    \item 数据有噪声
    \item 模型复杂度太高
\end{itemize}

\begin{figure}[H]
    \centering
\includegraphics[width=0.3\columnwidth]{sl/2.png}
\end{figure}

\paragraph{风险和经验风险}
风险最小化$\alpha^*=\arg\min_{\alpha \in \Lambda} R(\alpha)$,依靠损失函数、概率测量、参数空间。经验风险最小化$\alpha_N=\arg\min_{\alpha \in \Lambda} R_\text{emp}(\alpha)$,依靠损失函数、训练数据、参数空间。

\paragraph{一致学习} 因为大数定理,所以风险和经验风险能够足够接近。前面$R_\text{emp}(\alpha_N) = 0$的最大样本数$N$点就是VC维的大小,代表100\%的拟合,也就是最大能打散的样本数,
%VC维也叫做函数集的熵。
\begin{figure}[H]
    \centering
\includegraphics[width=0.7\columnwidth]{sl/3.png}
\end{figure}

举个特殊例子,如果参数空间只有一个参数$\alpha_0$,那么根据Hoeffding不等式(且采用0-1 loss)有:
$$P(R(\alpha_0) - R_\text{emp}(\alpha_0)) \le \exp(-2N\epsilon^2),$$
或者说,至少以$1-\eta$的概率有:
$$R(\alpha_0) \le R_\text{emp}(\alpha_0) + \sqrt{\frac{-\log \eta}{2N}},$$
那么更加一般的情况下,假设有M个有限参数,我们至少以$1-\eta$的概率有:
$$
\sup_{\alpha \in \Lambda}(R(\alpha) - R_\text{emp}(\alpha)) \le \sqrt{\frac{1}{2N}(\log M - \log \eta)},
$$
但是有些实用的学习模型有无限数量的参数!

\paragraph{函数集的熵}
考虑二分类$f(x|\alpha) \in \{-1,+1\}$,当我们给定一个数据集和一个参数空间时有多少种可能的结果?定义:
$\mathcal{N}^{\Lambda}(x_1, x_2, \ldots, x_N) = \mathrm{Count}\{(f(x_1|\alpha), \ldots, f(x_N|\alpha)) | \alpha \in \Lambda\},$
注意,我们同样可以根据损失函数(0-1 loss)来定义这个数字,并且这个数字是相等的。

我们定义熵为$H^\Lambda(N) = \mathbb{E}_{x\in \mathcal{X}} \log(\mathcal{N}^\Lambda)$,退火熵为$H_\mathrm{ann}^\Lambda(N) = \log(\mathbb{E}_{x\in \mathcal{X}} \mathcal{N}^\Lambda) > H^\Lambda(N)$,生长函数为$G^\Lambda(N) = \log(\sup_{x\in \mathcal{X}} \mathcal{N}^\Lambda)$。

理论1:一致学习\\
当且仅当$\lim_{N\to \infty} \frac{H^\Lambda(N)}{N} = 0$,满足$\lim_{N\to \infty} P(|\sup_{\alpha \in \Lambda}(R(\alpha) - R_\text{emp}(\alpha))| > \epsilon) = 0, \forall \epsilon > 0$。

理论2:快速收敛\\
如果$\lim_{N\to \infty} \frac{H_\mathrm{ann}^\Lambda(N)}{N} = 0,$那么$P(R(\alpha_N) - R(\alpha^*) > \epsilon) < \exp(-c\epsilon^2N), \forall \epsilon > 0, \forall N > N_0$

理论3:
当且仅当$\lim_{N\to \infty} \frac{G^\Lambda(N)}{N} = 0,$那么学习是一致的而且收敛是快速的,不管$p(x)$是什么。

\paragraph{VC维度}
我们可以进一步证明任何生长函数都要么满足$G^\Lambda(N) = N\log 2$,或者$G^\Lambda(N) \le h(\log (N/h) + 1)$,其中$h$是个整数且有$G^\Lambda(h) = h \log 2, G^\Lambda(h+1) < (h+1)\log 2$,$h$就是VC维度。VC维度的就是函数集可以打散的最多样本的数量$h$。

在$D$维空间的线性分类器中,其VC维为$D+1$。此时VC维与其参数的数量一样。

拥有可以调节频率的$\sin()$的模型,其VC维是无穷的!

SVM需要考虑一个受限的线性分类器:$$y=\left\{\begin{aligned} 
+1, \text{if}\ w^Tx+b\ge \Delta, \\
-1, \text{if}\ w^Tx+b\le \Delta,
\end{aligned}\right.$$的VC维是$h \le \min(D, [\frac{R^2}{\Delta^2}]) + 1$,$R$是能覆盖所有数据点的超球体的半径,小于线性分类器,其拟合能力比较弱。

直观地说,VC维度表征了一个函数集的强大程度。无限VC维度意味着,对于特定数据集,函数集总是可以实现等于0的经验风险,即 ERM不提供信息。严格地,VC维定义了风险的边界。
$$
\epsilon = 4\times \frac{h(\log(2N/h)+1) - \log(\eta/4)}{N},
$$
$$
\sup_{\alpha \in \Lambda}(R(\alpha) - R_\text{emp}(\alpha)) \le \sqrt{\frac{\epsilon}{4}},
$$
对于有限的参数空间,我们有$\epsilon = 2 \times \frac{\log M - \log \eta}{N}$

VC维度描述了无限函数集的“有效体积”。
但是许多重要分类器(决策树、神经网络)的VC维度尚不清楚,
VC维度分析不适用于非参数学习(例如 k-NN)。

\paragraph{结构风险最小化}
SRM试图最小化ERM和置信区间的总和。SRM是正则化的一个具体例子
$$\min_\alpha R_\text{emp}(\alpha) + \lambda C(\alpha)$$
其中,正则化项控制了模型的复杂度。正则化首先被提出来处理病态问题。有几种不同的方式来解释/执行正则化:贝叶斯、偏差-方差均衡、最小描述长度(MDL)。

\paragraph{贝叶斯作为正则化手段}
考虑概率建模$q(y|x)$
,损失函数是对数损失$-log q(y_i|x_i)$。
最小化经验风险就是极大似然,最大化后验$p(\alpha | \{x_i, y_i\}_{i = 1,\ldots, N})$,
需要先验$p(\alpha)$。
因此,正则化项等价于$-\log p(\alpha)$

贝叶斯与SRM对比:1)贝叶斯需要先验信息。
2)SRM不要求真实模型位于假设空间内。

\paragraph{偏差-方差均衡作为正则化手段} ERM没有考虑数据集的方差,这是类似于正则项的地方。但是偏差-方差均衡难以实现。

\paragraph{最小描述长度作为正则化手段} 最小描述长度(MDL)原则:最佳模型是在给定数据集上能够达到最小描述长度的模型。MDL考虑了模型存储的成本,相当于正则化项。MDL需要适当的编码规则;但是理想的编码与概率有关。

\paragraph{精度、召回率、AUC}
对于二元分类,我们通常希望区分不同类型的错误。
$$\text{Precision} = \frac{\text{TP}}{\text{TP} + \text{FP}}$$
$$\text{Recall} = \frac{\text{TP}}{\text{TP} + \text{FN}}$$
$$\text{F1-value} = \frac{2\times\text{Precision}\times\text{Recall}}{\text{Precision} + \text{Recall}}$$
最理想情况下,FN=FP=0。FP表示假阳性(误报),FN表示未检出的阳性(漏报)。

还有一个AUC,它是ROC (receiver operating characteristic) curve的曲线下面积。ROC曲线的X轴是$\frac{\text{FP}}{\text{FP} + \text{TN}}$,Y轴是$\frac{\text{TP}}{\text{TP} + \text{FN}}$(Recall)。随机的二分类器AUC为0.5,理想的为1。

\paragraph{数据集划分} K-fold交叉验证

\section{非参数学习}
许多统计学习方法假设了个模型,其学习过程就是解出或者估计出模型的参数。非参数学习没有明面上的模型,有时被称为基于实例/记忆的学习。
\paragraph{Parzen Window/核密度估计}
求解(概率)密度估计问题,给出一组样本$x_1,\ldots,x_N$,求$p(x)$。PW用核函数(需要满足非负、积分为1,比如说,高斯核)求和来估计:
$$\hat{p}(x) = \frac{1}{N}\sum^N_{i=1}K(x|x_i),$$
其实PW和直方图很像,一个是经验概率密度函数,一个是用核函数的相加估计真正的PDF。那么我们也需要一个超参数$h$用于控制窗口大小:
$$
\begin{array}{l}
K_h(x) = \frac{1}{h}K(\frac{x}{h}) \\
\hat{p}_h(x) = \frac{1}{N}\sum^N_{i=1}K_h(x|x_i),
\end{array}
$$
窗越小,越容易过拟合。

\paragraph{$k$-NN}看空间关系最近的$k$个样本,(简单/加权)投票决定样本归属哪一类。设$\mathcal{N}(x_i)$是$x_i$的邻居。
$$
\begin{array}{ll}
\text{分类} & \hat{y} = \mathrm{sign}(\sum_{x_i \in \mathcal{N}(x_i)} y_i) \\
\text{回归} & \hat{y} = \frac{1}{|\mathcal{N}(x_i)|} \sum_{x_i \in \mathcal{N}(x_i)} y_i \\
\end{array}
$$
1-NN对噪声太过敏感。用偏差-方差分解考察k-NN回归$\mathbb{E}[(\hat{y} - y)^2]$,发现同样有偏差-方差-噪声三项。$k\uparrow,\text{var} \downarrow,\text{bias} \uparrow$

\paragraph{稀疏编码} 目的是要求解:
$$x = \sum^N_{i=1} \alpha_ix_i,\text{ s.t. } \|\alpha\|_0 \le k,$$
可以放宽以处理数据中的噪声或损坏。

\section{无/半监督学习}
有监督学习意图找出数据之中的关系以解决预测问题。无监督学习意图发现数据中的模式,以对数据进行描述:比如关联分析(啤酒尿布,什么属性是互相关联的)、聚类(分类的弱化版)、异常检测(数据是否是正常的)、降维(避免维度诅咒)等等

\paragraph{维度诅咒}
\begin{enumerate}
    \item 在高维空间,距离无法区分 
    \item 最近的邻居都在很远的地方
\end{enumerate}
为什么?因为高维空间非常大,样本很稀疏。

\paragraph{K-means} 是Prototype-based clustering的代表。每个聚类都有个原型,样本距离原型的距离决定了样本的类别。
\begin{algorithm}[H]
\caption{K-means算法}
\label{alg:K-means}
\begin{algorithmic}[1]
\Require dataset $\{x_1, \ldots, x_N\}$, number of clusters $k$
\Ensure clusters $q(x_i)\in\{1, \ldots, k\}$
\State Initialize centroids $\{c_1, \ldots, c_2\}$
\Repeat
\For{$i = 1, \ldots, N$}
\State $q(x_i) \leftarrow \arg \min_j |x_i - c_i|$
\EndFor
\For{$i = 1, \ldots, k$}
\State $c_j \leftarrow \mathrm{mean}(x_i|q(x_i) = j)$
\EndFor
\Until{centroids do not change}
\end{algorithmic}
\end{algorithm}
通常来说这个算法会收敛。

中心点(也就是前面的prototype)也被称为码字,组成码本。k-means其实是解决
$$\min_{q,\{c_j\}} |x_i - c_{q(x_i)}|^2$$
k-means算法启发式地交替更新$q$和$\{c_j\}$。这是贪心的,不能保证得到全局最优解。经常受到离群点、不同大小的簇、不同密度的簇、不规则形状的影响。一个办法是过分割(增大$k$),然后再后处理。

\paragraph{高斯混合模型}
高斯混合模型是一种基于分布的聚类方法。每一个簇代表一个单模态的分布。计算后验概率以决定样本属于哪一个簇。在高斯混合模型中,每个分布是高斯分布。\begin{enumerate}
    \item 一维情况:$p(x) = \sum_{j=1}^k w_j\mathcal{N}(x|\mu_j, \sigma_j^2),\text{ where} \sum_j w_j = 1,$
    \item 多维情况:$p(x) = \sum_{j=1}^k w_j\mathcal{N}(x|\mu_j, \Sigma_j),\text{ where} \sum_j w_j = 1,$
\end{enumerate}
如果我们知道了GMM的参数,那么我们可以这样计算后验概率(也叫响应度):
$$p(q(x_i) = j) = \gamma_{ij} = \frac{w_j\mathcal{N}(x|\mu_j, \sigma_j^2)}{\sum_{j=1}^kw_j\mathcal{N}(x|\mu_j, \sigma_j^2)},$$
然后有$q(x_i) = \arg\max_j \gamma_{ij},$
所以说,我们怎么样估计GMM的参数$\theta = \{w_j, \mu_j, \Sigma_j | j = 1, \ldots k\}$呢?

问题就是要最大化$\prod_{i=1}^N \sum_{j=1}^k w_j \mathcal{N}(x_i|\mu_j, \Sigma_j)$,我们可以借鉴K-means的算法。首先初始化参数,然后计算相应的相应度,第三步更新参数。上面两个步骤交替进行,直到模型收敛。

记$\gamma_{ij} = p(q(x_i) = j)$:$$
\begin{aligned}
    w_j = \frac{\sum_i \gamma_{ij}}{N} \\
    \mu_j = \frac{\sum_i \gamma_{ij}x_i}{\sum_i \gamma_{ij}} \\
    \Sigma_j = \frac{\sum_i \gamma_{ij}(x_i - \mu_j)^T(x_i - \mu_j)}{\sum_i \gamma_{ij}} \\
\end{aligned}
$$

\paragraph{EM算法}
引入隐变量$z_i \in \{1, \ldots, k\}$,代表$x_i$对应的正确的簇,那么要最大化$\prod_i p(x_i, z_i|\theta) = \prod_i \prod_j (w_j\mathcal{N}(x_i|\mu_j, \Sigma_j))^{\mathcal{I}(z_i =j)}$,或者$\sum_i \sum_j \mathcal{I}(z_i =j)\log(w_j\mathcal{N}(x_i|\mu_j, \Sigma_j))$

EM算法分为交替的两步:E-step: 给定 $\theta^t$, 通过消除潜在变量来计算目标函数的期望值。 $\gamma_{ij}$ 是 $\mathcal{I}(z_i = j)$ 的期望, 所以我们有 $\sum_i\sum_j \gamma_{ij} \log(w_j\mathcal{N}(x_i|\mu_j, \Sigma_j))$

M-step: 最大化目标函数的期望值以找到新的参数估计 $\theta^t$. 然后我们可以导出GMM方程。

\begin{algorithm}[H]
\caption{EM算法}
\label{alg:EM}
\begin{algorithmic}[1]
\Require $\hat{\theta} = \max_\theta p(X,Z|\theta)$, where $Z$ is unobserved
\State $t \leftarrow 0$, initialize $\theta^0$
\Repeat
\State Given $\theta^t$, calculate the expectation of $\log p(X, Z|\theta)$ with eliminating $Z$, i.e. $Q(\theta, \theta^t) = \mathbb{E}_{Z\sim p(Z|X, \theta^t)}\log p(X, Z|\theta)$
\State $\theta^{t+1} \leftarrow \arg \max_{\theta} Q(\theta, \theta^t)$
\Until{convergence}
\State $\hat{\theta} = \theta^{t+1}$
\end{algorithmic}
\end{algorithm}

EM是一种贪婪算法,它肯定会收敛,但不能确保全局最优。
设置不同的初始值以逃避局部最优。我们可能无法最大化期望(即Q函数);相反,增加Q函数(例如通过梯度上升)是可以的;如果Q函数不容易最大化,这可能是有帮助的。

\paragraph{不同的聚类方法}
基于密度的聚类:Mean-shift: 局部密度的均值来替代;DBSCAN:对于每个点,如果邻点的数目小于一个阈值,那么这个点就是噪声。基于连通性的聚类: 基于图的聚类。合并聚类:自底向上聚类。分层聚类:自顶向下聚类。

\paragraph{PCA}
给出一个$x\in \mathbb{R}^D$,我们想要找到一个矩阵$P \in \mathbb{R}^{K \times D}$,其中$K < D$,然后我们可以用这个矩阵$P$降维$x$:$y = Px, y \in \mathbb{R}^K$。

第一步是$\bar{x} = \frac{\sum_i x_i}{N}$,令$X = \begin{pmatrix}
    (x_1 - \bar{x})^T \\
    \cdots \\
    (x_N - \bar{x})^T
\end{pmatrix}$,
第二步是$C = X^TX = U\Lambda U^H$,这是个协方差阵,可以找出其特征值和特征向量,第三步是选择最大的K项特征值,并且选择对应的特征向量组成$P$,$y_i = P(x_i - \bar{x})$

\paragraph{核PCA}
对于非线性,使用核函数实现。$\Phi = \begin{pmatrix}
    \phi_i^T \\
    \cdots \\
    \phi_N^T
\end{pmatrix}$,其中$\phi_i = \phi(x_i)$。我们关心的是$K = \Phi\Phi^T$。

\paragraph{流形学习}
流形是一种拓扑空间,它在每一点附近都与欧几里得空间相似。
一维流形包括线和圆,但不包括"8"。
二维流形也被称为曲面,如球体。
流形的内在维度可以低于它的驻留空间。
流形学习是为了从高维数据中识别这种低维结构。
对于流形学习,用测地线距离代替高维欧几里得距离。

\paragraph{ISOMAP} 
与PCA相似,PCA寻求尽可能保留数据的欧几里德距离,ISOMAP寻求尽可能保持数据的测地线距离。
在ISOMAP中,测地线距离被定义为图上的最短距离,其中图由每个点的最近邻居构建,可通过多维缩放(MDA)解决。

\paragraph{Locally linear embedding} 
LLE寻求尽可能保持局部线性关系。首先优化
\begin{equation}
\arg \min_W \sum_i |x_i-\sum_{x_j \in N(x_i)} W_{ij}X_j|^2, s.t. \sum_j W_{ij} = 1
\end{equation}
其中如果$x_j \notin N(x_i)$,$W_{ij}=0$, 然后得到低维的表示:
\begin{equation}
\arg \min_{y_1,...,y_n} \sum_i |y_i - \sum W_{ij}y_j|^2
\end{equation}

\paragraph{半监督学习} 分类vs聚类:分类擅长预测正确的类别,但是需要大量数据标注;聚类能够分类数据,不等同于准确的类别,不需要标注。
监督学习vs半监督学习:监督学习的一个实际困难是缺乏准确的标签,半监督学习尝试使用未标记的数据和标记的数据,包括转导学习(不建立模型,只对未标记的数据进行预测)。

对于生成模型:
标注数据为$p(x,y)$,未标注数据为$p(x)$,后者对前者的估计有帮助。
对于判别模型:
1)聚类假设:如果两个点属于同一个聚类,它们很可能属于同一类。
2)密度假设:决策边界应位于分隔高密度区域的低密度区域。

\paragraph{PageRank} 基于Random Walk,用来定义网页的相对重要度的。
\begin{algorithm}[H]
\caption{PageRank算法}
\label{alg:PageRank}
\begin{algorithmic}[1]
\Require A graph of webpages and hyperlinks
\Ensure Relative importance values of all webpages
\State $t \leftarrow 0$, initialize $r_i^0$ uniformly
\Repeat
\State $\forall j, r_j{t+1} \leftarrow 0$
\State $\forall i,j, \mathrm{if}\ w_{ij} \neq 0, r_j^{t+1} = r_j^{t+1} + \frac{w_{ij}}{\sum_k w_{ik}}r_i^t$
\State $\forall j, r_j^{t+1} \leftarrow \beta r_j^{t+1} + \frac{1-\beta}{N}$
\Until{convergence}
\end{algorithmic}
\end{algorithm}

\section{组合学习}
建立多个个体/基础学习器,然后将它们组合起来。当这些基础学习器好而不同的时候,组合学习很有效果。但是实际中,基础模型们很难做到独立。我们希望每个基础模型尽可能不同,但是多元化和性能是冲突的。有两种形式的组合学习:
\begin{enumerate}
    \item Boosting: 每个基础模型都顺序地训练,整体模型更加关注模型之前处理得不太好的样本。
    \item Bagging: 每个模型都是独自、并行地训练,整体模型尝试使每个基础模型的训练数据多样化。
\end{enumerate}

\paragraph{Boosting}
$$f_m(x) = f_{m-1}(x) + \alpha_mG_m(x),$$
其中$G_m(x)$是基础模型,那么整体模型是
$$f(x) = \sum_m \alpha_m G_m(x)$$

\paragraph{Boosting回归树}
$$f(x) = \sum_m \alpha_m T(x, \theta_m),$$
其中,$T(x,\theta_m)=\left\{\begin{array}{ll}
    c_{m1}, & \text{if }x \le t_m, \\
    c_{m2}, & \text{if }x > t_m,
\end{array} \right.$是决策树桩。我们利用残差训练该模型,算法如下:
\begin{algorithm}[H]
\caption{Step-wise learning algorithm}
\label{alg:step-wise}
\begin{algorithmic}[1]
\Require $\{x_n, y_n\}$
\Ensure $\{\alpha_m, \theta_m\}$
\For{$m=1, \ldots, M$}
\State Calculate residue ${r_n}^{(m)} = y_n - f_{m-1}(x_n)$
\State Fit the residue with decision stump $T(x,\theta_m)$
\State Set $\alpha_m = 1$
\State Update the model $f_m(x) = f_{m-1}(x) + \alpha_mT(x, \theta_m)$
\EndFor
\end{algorithmic}
\end{algorithm}

\paragraph{\textit{Adaptive Boosting}}
分类模型,目标是优化:
$$\sum_{n=1}^N \mathcal{L}(y_n, f(x_n)) = \sum_{n=1}^N \exp(-y_n \times f(x_n)),$$
最小化$$\sum^N_{n=1} \mathcal{L}(y_n, f(x_n)) = \sum_{n=1}^N \exp(-y_n \times f(x_n)),$$也就是:
$$
\sum_{n=1}^N \exp(-y_n \times [f_{m-1}(x_n) + \alpha_mG_m(x_n)]),
$$

\begin{algorithm}[H]
\caption{AdaBoost Algorithm}
\label{alg:AdaBoost}
\begin{algorithmic}[1]
\Require $\{x_n, y_n\}$
\Ensure $f(x) = \sum_m \alpha_mG_m(x)$
\For{$m=1, \ldots, M$}
\State Calculate weights of samples: $w_{mn} = \exp(-y_n \times f_{m-1}(x_n))$ and then
normalize $w_{mn} \leftarrow \frac{w_{mn}}{\sum_n w_{mn}}$\begin{itemize}
    \item $m=1$, then $w_{1n} = 1/N$
    \item $w_{(m+1)n} \propto w_{mn}\exp(-y_n\alpha_mG_m(x_n))$
\end{itemize}
\State $G_m(x)$ is to minimize $e_m = \sum_{n=1}^N w_{mn}\mathbb{I}(y_n \neq G_m(x_n))$
\State $\alpha_m = \frac{1}{2} \log \frac{1-e_m}{e_m}$
\EndFor
\end{algorithmic}
\end{algorithm}

如果一个样本被正确分类,其权重就会减少$exp(-\alpha_m)$。
如果一个样本被错误地分类,它的权重会增加$exp(-\alpha_m)$。
因此,下一个基础分类器将专注于错误分类的样本。

\paragraph{Bagging = bootstrap aggregating} 通过自举采样生成多个数据集。生成$M$个数据集,用每个数据集来训练一个模型。然后对它们进行平均:$f(x) = \frac{1}{M} \sum_m G_m(x)$,可以并行学习。

\section{决策树}
树模型由一组条件和一组基本模型组成,以树的形式组织起来。每个内部节点都是一个针对输入属性的条件——对输入空间的划分。每个叶子节点就是一个基本模型,回归时最简单为一个常数,分类时最简单为一个类别。

\paragraph{构建决策树} 是个NPH问题。所以穷尽搜索不可行,我们应该用启发式方法,如下:
\begin{algorithm}[H]
\caption{Hunt's algorithm}
\label{alg:hunt}
\begin{algorithmic}[1]
\Require A set of training data $\mathcal{D} = \{x_n, y_n\}$
\Ensure A classification tree or regression tree T
\Function{HuntAlgorithm}{$\mathcal{D}$}
\If{$\mathcal{D}$ need not or cannot be divided}
\State \Return a leaf node
\Else
\State Find an attribute of $x$, say $x_d$, and decide a condition $g(x_d)$
\State Divide $\mathcal{D}$ into $\mathcal{D}_1, \mathcal{D}_2, \ldots$, according to the output of $g(x_d)$
\State $T_1 = \textproc{HuntAlgorithm}(\mathcal{D}_1), T_2 = \textproc{HuntAlgorithm}(\mathcal{D}_2), \ldots$
\State Let $T_1, T_2, \ldots$ be the children of $T$
\EndIf
\State \Return T
\EndFunction
\end{algorithmic}
\end{algorithm}

\paragraph{纯洁度/不纯度} 描述一个集合容易/不容易分为一类的程度。下面是几种不纯度(越小越好)测量方法,用$p_i$表示类$i$的占比:
\begin{enumerate}
    \item Entropy: $-p_0\log p_0 - p_1\log p_1$
    \item Gini index: $1-p_0^2-p_1^2$
    \item Misclassification error: $\min(p_0, p_1)$
\end{enumerate}
我们还要找到怎么样决定对一个属性进行划分。当然是纯洁度增益越大越好。这里给出了三个计算增益的方法,其中$H$是上面的熵,$G$是Gini:
\begin{enumerate}
    \item Information gain: $g = H(\mathcal{D}) - \sum_i \frac{|\mathcal{D}_i|}{|\mathcal{D}|}H(\mathcal{D}_i)$
    \item Information gain ratio: $gr = \frac{g}{-\sum_i \frac{|\mathcal{D}_i|}{|\mathcal{D}|}\log \frac{|\mathcal{D}_i|}{|\mathcal{D}|}}$
    \item Gini index gain: $gig = G(\mathcal{D}) - \sum_i \frac{|\mathcal{D}_i|}{|\mathcal{D}|}G(\mathcal{D}_i)$
\end{enumerate}

\paragraph{树的剪枝}
采用算法\ref{alg:hunt},我们可以构建一个预测尽可能准确的树,但是可能发生过拟合。有两个方案控制树的复杂度:
\begin{enumerate}
\item 早停止:停止划分,如果增益小于阈值,或者树太深、集合太小
\item 树剪枝:从树中移除一些分支,以降低总体的误差$C_\alpha(T) = C(T) + \alpha|T|$,其中$C(T)$是经验风险(比如说预测错误率),$|T|$是树的复杂度(比如说树的高度)。
\end{enumerate}

\paragraph{回归决策树}
最简单的情况树每个叶子节点代表一个常数。每次寻找一个属性并且选择一个划分条件,最小化误差:
$$\min_{d, t, c_1, c_2}\left[\sum_{x_{id} \le t}(y_i - c_1)^2 + \sum_{x_{id} > t}(y_i - c_2)^2\right]$$
最终这个回归树是个分段常函数。

\paragraph{回归决策树和boosting方法的等价}
Hunt算法:“分而治之”,条件+基础模型。Boosting:基础模型的线性组合。本质上是一样的,得到的东西也一样。

\paragraph{树模型的实现}
\begin{itemize}
    \item ID3:用information gain
    \item C4.5:用information gain ratio
    \item CART:用Gini index (分类) 或者quadratic cost(回归,上面有说),只用2路划分。
\end{itemize}
根据$C_\alpha(T) = C(T) + \alpha|T|$,逐渐增大$\alpha$以获得不同的树,然后用交叉验证寻找最佳的$\alpha$。

\paragraph{随机森林}
- 决策树和集合学习的结合 \\
- 根据袋法,首先生成多个数据集(bootstrap samples),每个数据集都会产生一个树模型\\
- 在构建树的过程中,在分割时考虑一个随机的特征子集

\section{概率图模型}
\paragraph{生成模型和判别模型}
生成模型是学习估计$p(x,y)$,也就是$x$和$y$的联合分布。判别模型是学习、估计$p(y|x)$,或者更加简单的$y=f(x)$。我们只需要建模$x$与$y$之间的关系。在分类问题中,生成模型通常比判别模型更加难,就像写作比阅读难一样。概率图模型通过分解简化联合分布 

\paragraph{朴素贝叶斯} 分类模型。
朴素贝叶斯用数据直接估计$p(y)$和$p(x|y)$,以求得$p(y|x)$。
$$p(y|x) \propto p(y)p(x|y) = p(y)\prod_{i=1}^D p(x_i|y)$$

\paragraph{拉普拉斯平滑}
为了让$p(y)$和$p(x_i|y)$不等于0,有必要进行平滑操作,假设$N()$是数据集中满足某种条件的数据的个数,$C()$是某一维数据的类别数目:
$$p^\prime(x_i = a|y = b) = \frac{N(x_i = a, y = b) + \alpha}{N(y = b) + \alpha \times C(x_i)}$$

解释:为什么频率等于概率——满足最大似然估计,为什么加上平滑——加上了狄利克雷分布作为先验,满足贝叶斯的最大后验估计。

\paragraph{贝叶斯网络} 是有向无环图。若一节点$d$有来自点$(a,b,c)$的进入该节点的边,那么:
$p(a,b,c,d) = p(d|a,b,c)p(a,b,c)$

\paragraph{D-划分和条件独立}
假设我们要考虑$p(A,B|C)$ 其中$A$,$B$,$C$是不相交的随机变量集,将$C$的节点标记为“已知”,对于每一对$a \in A$、$a \in B$,找出从$a$到$b$的所有可能路径,并判断这条路径是否被阻塞。无论箭头方向如何,路径都由连续的边组成。
\begin{enumerate}
\item 如果一个节点是公共父节点或链中节点,并且该节点是\textbf{已知}的,则路径被阻塞。
\item 如果一个节点是公共子节点,并且该节点及其所有后代都是\textbf{未知}的,则路径被阻塞。
\end{enumerate}
如果所有的路径都被阻塞了,那么$A \perp \!\!\! \perp B | C$,否则$A \not \! \perp \!\!\! \perp B | C$

\paragraph{马尔可夫盘}
使用条件独立,我们可以证明:$$p(A|\bar{A}) = p(A|\partial A),$$其中,$\bar{A}$是$A$的补集;$\partial A$是$A$的邻域,包括:父节点,子节点,子的其他父节点。这样的邻域叫做马尔可夫盘。

\paragraph{马尔可夫随机场} 无向图模型,没有直接的条件分布。条件独立的要点是若去掉$C$中节点,$A$和$B$之间没有路径相通,那么$A \perp \!\!\! \perp B | C$。其马尔可夫盘就是直接的邻居。

马尔可夫随机场定义了相关,但不是分布。
- 如果$x_i$和$x_j$不相连,那么我们就不需要定义$p(x_i, x_j |\{x_i , x_j\})$ ,因为它们是条件独立的。我们只需要考虑连接的节点。团:一个节点集合,其中任何两个节点都是相连的。最大团:在图中具有最大可能大小的团。联合分布只能在团上定义,最终在最大团上有个分布。实践中,我们常常用指数族作为联合分布。

\paragraph{转换贝叶斯网络到马尔可夫随机场}
有向边改为无向边,有共同子节点的父节点间加上无向边。转换后,每个节点的马尔可夫盘保持不变,有一些信息丢失。一个马尔可夫随机场可对应多个贝叶斯网络,也可能无对应。

\paragraph{因子图}
因子用黑方块表示,每个因子是个函数或者概率表达式,其参数是黑方块所连接的节点。每个节点也必定连接相关的因子。

\begin{figure}[H]
    \centering
\includegraphics[width=0.7\columnwidth]{factor.png}
\end{figure}

\paragraph{信念传播:和-积算法}
有两种信息:从变量到因子,从因子到变量。

从变量到因子:$\mu_{x_m \to f_s}(x_m) = \prod_{l \in \mathrm{ne}(x_m) \backslash f_s} \mu_{f_l \to x_m}(x_m)$,也就是$f_l$是除$f_s$以外的所有因子。

从因子到变量:$$
\begin{aligned}
&\mu_{f_s \to x}(x) = \\
&\sum_{x_1}\! \cdots\! \sum_{x_M}\! f_s(x,x_1,\! \ldots,\! x_M)\! \prod_{m \in \mathrm{ne}(f_s)\! \backslash\!  x}\! \mu_{x_m\! \to\! f_s}(x_m)
\end{aligned}
$$,也就是$x_m$是除$x$以外的所有因子。

对于连续变量,和-积算法仍然有效,用PDF代替了概率分布。
如果因子图是一棵树(即没有循环),和-积算法是精确的。
如果因子图包含循环,可以使用循环信念传播法。
需要决定一个消息传递时间表。
不一定能收敛。
在很多情况下,我们退而求其次,进行近似推理。

\section{深度学习}
\paragraph{M-P神经元模型}
最基本的神经网络单元,连接具有权值,有激活函数。
$$y = f(\bm w^T\bm x - \theta),$$
其中,激活函数可以是sign或者sigmoid。感知机不能处理线性不可分的分类问题。

\paragraph{MLP}
多层感知机,中间层被称为隐藏层。有非线性的激活函数。

\paragraph{Feedforward Network}
每层都是全连接层,没有同层连接(RNN)、跳层连接(ResNet)。输入层没有激活函数,隐藏层和输出层一般有激活函数。

\paragraph{同层连接}
RNN带来了同层的连接,使得每个timeslot可以使用之前的timeslot的输出/隐藏状态。

\paragraph{跳层连接}
ResNet引入了残差连接,DenseNet引入了稠密连接。

\paragraph{BP和梯度下降}
和前面没有什么不同。值得一提的是批处理——这是在随机梯度下降和传统梯度下降间的tradeoff。

\paragraph{动量}
动量就是上一次梯度下降的梯度信息,有一阶和二阶(就是一阶梯度信息的平方)之分。如果在梯度下降中加入动量,通常能加快训练速度,提升训练精度。几乎所有的优化器都用了动量

\paragraph{避免过拟合} 只要有一个隐藏层(保证足够多的神经元),一个FFN就可以在任意精度上拟合任意的连续函数。所以神经网络非常容易过拟合。避免过拟合有以下方法:
\begin{enumerate}
\item 使用验证集,以早停止
\item 使用正则化
\item Dropout(丢弃神经元)和DropConnect(丢弃神经元的连接边)
\end{enumerate}

\paragraph{非监督学习}
{SOM}
竞争学习:在输出层,只有一个神经元被激活,其他神经元被抑制。(输出类似One-Hot)
输出层是 2D。可用于降维。

{Hopfield network}
神经元完全相互连接。通常,约束是与自身无连接且连接是对称的。每个神经元的状态是二进制的(-1 或 1)。它考虑了联想记忆。

{Boltzmann machine}
类似于 Hopfield 网络,但区分可见和隐藏单元。

{Restricted Boltzmann machine}
基于能量的模型。只允许可见和隐藏神经元之间的连接,是一个二分图。

{DBN}
可以用于降维。DBN 也代表了“自动编码器”的策略,它试图从无监督学习中受益。

{PixelCNN}
用概率来刻画图像

{Variational auto-encoder}
将自动编码器转换为概率框架。

{GAN}
拥有生成器和判别器。

\paragraph{判断题}
对:
1.使用一组测量值的算术平均值等效于对这组测量值求解一个最小二乘问题 2.将 p(x|y)视作 y 的
函数,则称之为似然函数 3.如果数据的个数 N 远远小于基函数的个数 M,则使用等效核函数可
以获得计算效率的提高 4.使用 L1-norm 的优点是能够获得稀疏的解向量 5.LASSO 回归也可以
解释为假设参数先验服从 Laplace 分布 6.Logistic 回归可以理解为一种极大似然估计的方法 7.KL 散度恒大于等于零 8.牛顿-拉夫逊法需要计算目标函数的 Hessian 矩阵 9.梯度下降法不能确保
找到目标函数的全局极小值 10.Fisher 线性判别分析,试图最大化类间距离、最小化类内方差 11.
感知机学习过程有时不收敛 12.Softmax 回归得到的分类面必然是线性的 13.在 SVM 训练过程
中,只要找出支持向量,就不用考虑其他的训练样本 14.所有的核函数一定能写成内积形式 15.
给定独立同分布的训练集和验证集,以最小均方误差为目标,训练一个线性回归模型(无基函数)。
如果在训练集和验证集上,该模型得到的均方误差都很大,并且两个均方误差基本一致,那么
该模型的偏差很大,该模型方差很小。引入一组基函数,重新训练线性回归模型。基函数的维度非常高,那么结果一般来说, 会使得模型的偏差变小,但方差不会变小。在训练目标中加入
正则化项,并给正则化项比较大的权重,那么结果一般来说,会使得模型的方差变小,但偏差
不会变小 16.使用正则化不一定能提高在验证集上的正确率 17.生长函数总是单调不减函数 18.
如果函数集 1 是函数集 2 的子集,那么 1 的 VC 维小于或等于 2 的 VC 维
错:
1.统计学习中一般不假设数据服从独立同分布 2.通过将约束优化问题转为其对偶形式,总是可
以获得原问题的最优解 3.凸优化问题只能有一个全局极小点 4.使用基函数是为了减少数据的维
度 5.正则化项的系数(λ)越大,则拟合的偏差越大、方差越大 6.线性分类,即规定分类器为一个
线性函数 7.Logistic 回归是一种非线性回归方法 8.一般而言,使用目标函数二阶微分的优化方法
是梯度下降法 9.梯度下降法可以确保找到目标函数的局部极小值 10.梯度下降法不能确保找到
目标函数的全局极小值,而牛顿-拉夫逊法可以 11.如果感知机学习过程不能收敛,可能是搜索
步长不合适,找到合适的搜索步长一定能收敛 12.对于线性可分的数据集,用 SVM 学习得到的
分类器与用感知机学习得到的分类器是完全一样的 13.使用硬间隔或者软间隔 SVM,必须先判
断数据集是否线性可分,可分就用硬间隔,不可分就用软间隔 14.对于软间隔 SVM,支持向量
指的是那些满足 $y_i(w·x_i+b)=1$ 的训练样本 15.核技巧比较适合于训练样本数量很多且维度很
低的情形 16.使用正则化以后,就不会出现过拟合现象了 17.正则化只控制经验风险,与模型要
学习的参数无关 18.使用正则化以后,验证时的损失函数也要相应改变 19.如果 VC 维是无限
大,说明函数集能够打散任意样本集 20.D 维空间中,线性支持向量机的 VC 维就是 D+121.非
参数监督学习方法没有显式参数,所以也不存在过拟合问题